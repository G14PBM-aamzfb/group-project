\documentclass{article}

\begin{document}
	\section{Introduction}
	
	In 1952 Alan Turing published his seminal work "The Chemical Basis of Morphogenesis" in which he described the development of complex patterns from originally uniform distributions of chemical substances. This process occurs through the catalytic interaction and simultaneous diffusion of two or more morphogens. Subsequently the idea has become a critically important foundation to models in many subfields of theoretical biology.
	
	Garfinkel et al. formulate a quantitative model of the morphogenesis of vascular mesenchymal stem cells based on reaction-diffusion equations describing the concentrations of \textit{bone morphogenetic protein 2} (BMP-2; a promoter of cell growth, i.e. the activator morphogen) and its inhibitor \textit{matrix carboxyglutamic acid protein} (MGP). Using numerical methods, they were able to simulate the evolution of their model system under different conditions and recreate a number of experimental observations, thereby applying the explanatory potential of the model to a real physiological mechanism.
	
	Crucially, they describe the importance of the ratio between both the diffusivity and the rate of decay of BMP-2 and MGP for the formation of trabecular patterns.
	
	\pagebreak
	
\end{document}