\documentclass{article}

\begin{document}
\section{Biology}

\subsection{MSC}

In the developing embryo, mesenchymal stem cells are pluripotent progenitor cells which divide many times and eventually differentiate through a series of separate and unique lineage transitions into a variety of end-stage phenotypes: cartilage, bone, tendon, ligament, marrow stroma, connextive tissues.
The term, mesenchyme, is derived from the Greek meaning “middle” (meso) “infusion” and refers to the ability of mesenchymatous cells to spread and migrate in early embryonic development between the ectodermal and endodermal layers.
This characteristic migratory, space-filling ability is the key element of all wound repair in adult organisms involving mesenchymal cells in skin (dermis), bone (periosteum), or muscle (perimysium).

Mesenchymal stem cells have the ability to form unique developmental structures, or, in adults to form repair blastemas that are able to achieve regenerative repair, a fact that makes their study crucial as it offers the potential for a new class of therapeutic technologies based on self-cell repair. 
Several important advances allow us to consider the possibility of using a patient’s own mesenchymal stem cells as starting material for tissue repair protocols.
Mesenchymal stem cells must exist to maintain the living organisms, just as hematopoietic stem cells must exist to support both red and white blood cell turnover. 

We might be able to isolate such human mesenchymal stem cells and place them in cell culture, where we could mitotically expand their numbers. Eventually, if we had enough of these cells, we could reintroduce them into the original donor in a manner that guaranteed that they would massively differentiate into a specific tissue, such as cartilage or bone, at a transplantation or repair site. Immunorejection would not be a problem because the donor and host would be one and the same.


\subsection{Morphogen \& Pattern Formation}


A morphogen is a substance that establishes a graded distribution and elicits distinct cellular responses in a dose-dependent manner that governs the pattern of tissue development in the process of morphogenesis or pattern formation. 
They function to provide individual cells within a field with positional information, which is interpreted to give rise to spatial patterns.
Morphogens can consist of cytoplasmic proteins, such as transcription factors that form a gradient by diffusion within a single cell or syncytium, or secreted signaling molecules that travel from cell to cell. 
In most cases, morphogens guide the generation of different cell types in a specific spatial order, usually by inducing unique transcriptional responses in a dose dependent manner.
Once the gradient has formed, cells must distinguish small differences in morphogen concentration and store this information even after the gradient has dissipated.

Pattern formation is the process by which a specialized cell, the fertilized egg, divides to give rise to seemingly identical cells. 
Over time and space, these cells take on distinct fates and eventually become organized into functional organ systems. 
Within a single organ, such as the heart, multiple classes of differentiated cells must synchronize their duties to ensure that the organ as a whole is functional. 
One of the unifying principles of development is the idea that embryogenesis proceeds via the iterative process of generating naïve fields of cells, and then providing cells within each field with unique positional information, which they then interpret to give rise to spatial patterns. 
Through this process, the embryo is sequentially subdivided, initially along the major body axes, and then into smaller, and more refined units such as organ primordia that are further partitioned and patterned	
\end{document}
